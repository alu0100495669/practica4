\documentclass[a4paper,12pt]{article}
\usepackage{graphicx}
\usepackage[spanish]{babel}
\usepackage[utf8]{inputenc}
\begin{document}
\title{Latex}
\author{Ignacio Fragoso Brito\\
	Técnicas Experimentales~\footnote{Universidad de La Laguna}
	}
\date{\today}
\maketitle

\begin{abstract}
  LaTeX es un sistema de composición de textos que está formado mayoritariamente
  por órdenes construidas a partir de comandos de TeX —un lenguaje «de nivel bajo», 
  en el sentido de que sus acciones últimas son muy elementales— 
  pero con la ventaja añadida de «poder aumentar las capacidades de LaTeX utilizando
  comandos propios del TeX descritos en The TeXbook».
  Esto es lo que convierte a LaTeX en una herramienta práctica y útil pues, a su facilidad de uso,
  se une toda la potencia de TeX. Estas características hicieron que LaTeX se extendiese rápidamente
  entre un amplio sector científico y técnico, hasta el punto de convertirse en uso obligado en comunicaciones
  y congresos, y requerido por determinadas revistas a la hora de entregar artículos académicos.
\end{abstract}

\end{document}
