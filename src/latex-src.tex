\documentclass[a4paper,12pt]{article}
\usepackage{graphicx}
\usepackage[spanish]{babel}
\usepackage[utf8]{inputenc}
\begin{document}
\title{Latex}
\author{Ignacio Fragoso Brito\\
	Técnicas Experimentales~\footnote{Universidad de La Laguna}
	}
\date{\today}
\maketitle

\begin{abstract}
  LaTeX es un sistema de composición de textos que está formado mayoritariamente
  por órdenes construidas a partir de comandos de TeX —un lenguaje «de nivel bajo», 
  en el sentido de que sus acciones últimas son muy elementales— 
  pero con la ventaja añadida de «poder aumentar las capacidades de LaTeX utilizando
  comandos propios del TeX descritos en The TeXbook».
  Esto es lo que convierte a LaTeX en una herramienta práctica y útil pues, a su facilidad de uso,
  se une toda la potencia de TeX. Estas características hicieron que LaTeX se extendiese rápidamente
  entre un amplio sector científico y técnico, hasta el punto de convertirse en uso obligado en comunicaciones
  y congresos, y requerido por determinadas revistas a la hora de entregar artículos académicos.
\end{abstract}

\includegraphics[width=0.5\textwidth]{imagen1.ps}

\section{Algo de historia}
Hablar de LaTeX en si es hablar del software libre y su idea principal que se puede modificar el software 
para adaptarlo a diferentes medios de trabajo y circunstancias.Pues bien fue Donald E. Knuth el creador de TeX,
lo que serviría de base a Leslie Lamport en 1984 para iniciar el trabajo de crear LaTeX, una serie de datos
que permitía utilizar de forma mas comoda la tipografia de Donald E. knuth llamada Tex; haciendo un parentesis
a Donald E. Knuth no le gustaba la tipografia que se presentaba en los volúmenes I, II, III de su obra.
El arte de programar ordenadores.En el tipio impulso de un hacker empezo a diseñar la tipografia Tex, 
lo cual empezo en 1978 pensando Donald en terminarla el mismo año, pero esto tardo 8 años cuando concluye 
en el año de 1985 lo cual no ha dado mas cambios sustanciales de version, la cual se quedo en la version 3
y cada nuevo cambio se hace en decimal. Por lo que la versión actual es la 3.1459 y Donald E Knuth a dicho
que la ultima modificación sera  una despues de su muerte, considerandose el programa concluido y todos los
bugs que se quedaran seran considerados como típicos del programa. Tex como se ha dicho es una tipografia 
muy estable y versatil, pero no incluye la mayoria de las lenguas conocidas con todo y sus caracteres adicionales,
al ser un codigo abierto este se puede modificar y para evitar que las modificaciones de Tex en diferentes idiomas
llevaran al lenguaje tipografico a una sima y fracmentacion del mismo Leslie Lamport desde 1984 implemento algunas
macros de TeX; siendo conocidas despues como LaTeX.
\subsection{Trabajo de Lamport}
El trabajo de Lamport  no ha parado en eso en 1989 Lamprot y otros desarrolladores iniciaron el "Proyecto "LaTeX3" 
que es una reestandarización completa de LaTeX, mediante una nueva versión que incluía la mayor parte de las 
extensiones adicionales que  se han estado haciendo con el tiempo; Este trabajo lo realizo Frank Mittlebach, 
Johannes Braams, Chris Rowley y Sebastian Rahtz junto con el propio Lesli Lamport hasta alcanzar el objetivo final
en el "Proyecto 3"  por lo que es actualmente conocido como LaTeX23 (o sea, "versión 2 y un poco más...").
Actualmente cada año se ofrece una versión nueva, aunque las diferencias son las minimas.

\section{Estructura interna}
Por otra parte la estructura interna del programa requiere de más detalles. Presentaremos algunos de las instrucciones
(etiquetas) que se deben de utilizar a la hora de formatear documentos elaborados en “LaTeX”. Proseguiremos
a mostrar la funcionalidad de las etiquetas más utilizadas para la elaboración del cuerpo de este tipo de
documento. Entre las etiquetas tenemos las siguientes: idioma, caracteres, párrafos.
\subsection{Idioma}
La configuración del idioma es muy importante. Por ejemplo, este tipo de documentos no aceptan la incorporación
de tildes ni de la “ñ”. Lo anterior es muy fácil de solucionar, con la debida configuración, podemos hacer que el
programa incorpore dichos símbolos sin muchos problemas.
\subsection{Caracteres}
Como todos los lenguajes, este tiene algunos caracteres especiales, los cuales no se pueden utilizar directamente,
es decir palabras propias del lenguaje. Mediante el correcto llamado se pueden utilizar sin ningún problema. Además
de soportar variados tipos, estilos y tamaños de letras. Esto permite crear documentos en los cuales se puede, tanto
diferenciar aspectos, como resaltarlos.
\subsection{Párrafos}
 Para el manejo de textos de contenidos extensos,este permite enmarcar diferencias que otros lenguajes no realizan.
Algunas de ellas son: alinear el texto, resaltar mediante el uso de cajas de textos, manejo de columnas, numeración,
etc.

\bigskip
 \begin{tabular}{|l|c|c|}
 \hline
 Nombre & Edad & Nota \\ \hline
 Pepe & 24 & 10 \\ \hline
 Juan & 19 & 8 \\ \hline
 Luis & 21 & 9 \\ \hline
 \end{tabular}

\end{document}
